\documentclass[UTF8]{ctexart}
\usepackage{amsmath}
\usepackage{graphicx}
\usepackage{geometry}
\usepackage{fancyhdr}


\title{Hello World Tex}
\author{张义举}
\date{\today}

%设置页边距
\geometry{papersize={20cm,30cm}}
\geometry{left=1cm,right=2cm,top=3cm,bottom=4cm}

%页眉页脚
\pagestyle{fancy}
\lhead{Yiju Zhang}
\chead{\today}
\rhead{zh1372337794@163.com}
\lfoot{}
\cfoot{\thepage}
\rfoot{}
\renewcommand{\headrulewidth}{0.4pt}
\renewcommand{\headwidth}{\textwidth}
\renewcommand{\footrulewidth}{0pt}

%段间距
\addtolength{\parskip}{.4em}

%行间距


\begin{document}
\maketitle
\tableofcontents

%-------------------------------
\section{你好郑州大学}
准备过程:包括对所要爬取的网站的分析、$E=mc^2$选取何种网络库及原因。实现过程:包括源程序的实现过程,代码要结构清晰,重点变量和重点功能等需要加上清晰的注释。实验结果:将所爬取的内容存储到文件中。并对所爬取到的数据进行简单清洗
\subsection{你好北核心教学区}
北核心教学区是我上课的地方
\subsubsection{你好东城}
东城是北京的中心。
\paragraph{天安门广场}
is in the center of Beijing!
\subparagraph{毛主席}
在天安门广场中心
\subsection{你好郑大}
郑大是宇宙中心

%-------------------------------
\section{插入数学公式}

\subsection{插入公式}
\paragraph{行内公式}
$Einstein 's  E=mc^2$
\paragraph{行间公式}
\subparagraph{没有标号}
\[Einstein 's  E=mc^2\]
\subparagraph{有标号}
\begin{equation}
Einstein 's  E=mc^2
\end{equation}
\paragraph{根式与分式}
$\sqrt{x}$, $\tfrac{1}{2}$
\[\sqrt{x},\]
\[\frac{1}{2}.\]

\subsection{插入运算符}
\subparagraph{普通运算符}
\[\pm \; \times \; \div \; \cdot \; \cap \; \cup \;
\geq \; \leq \; \neq \; \approx \; \equiv \]
\subparagraph{大型运算符}
\[\sum \; \prod \; \lim \; \int \]

\subsection{复合}
$ \sum_{i=1}^n i\quad \prod_{i=1}^n $
$ \sum\limits _{i=1}^n i\quad \prod\limits _{i=1}^n $
\[ \lim_{x\to0}x^2 \quad \int_a^b x^2 dx \]
\[ \lim _{x\to0}x^2 \quad \int\nolimits_a^b x^2 dx \]
\[ \iint\quad \iiint\quad \iiiint\quad \idotsint \]

\subsection{定界符(括号)}
\[ \Biggl( \biggl( \Bigl( \bigl( (x) \bigr) \Bigr) \biggr) \Biggr) \]
\[ \Biggl[ \biggl[ \Bigl[ \bigl[ [x] \bigr] \Bigr] \biggr] \Biggr] \]
\[ \Biggl\{ \biggl\{ \Bigl\{ \bigl\{ \{x\} \bigr\} \Bigr\} \biggr\} \Biggr\} \]
\[ \Biggl\langle \langle x \rangle \Biggr \rangle \]
\[ \Biggl\lvert x \Biggr\rvert \]
\[ \Biggl\lVert x \Biggr\rVert \]

\subsection{省略号}
\[ x_1, x_2, \dots , x_n\quad 1, 2, \cdots ,n\quad \vdots\quad \ddots \]

\subsection{矩阵}
\[ \begin{pmatrix} a&b\\c&d \end{pmatrix} \quad
\begin{bmatrix} a&b\\c&d \end{bmatrix} \quad
\begin{Bmatrix} a&b\\c&d \end{Bmatrix} \quad
\begin{vmatrix} a&b\\c&d \end{vmatrix} \quad
\begin{Vmatrix} a&b&c\\c&d&e\\f&g&h \end{Vmatrix} \]
Marry has a little matrix $ (\begin{smallmatrix} a&b\\c&d \end{smallmatrix} ) $.

\subsection{多行公式}
\paragraph{长公式}
\begin{multline}
x = a + b + c + {} \\
d + e + f + g
\end{multline}
\begin{multline*}
x = a + b + c + {} \\
d + e + f + g
\end{multline*}
\[ \begin{aligned}
x = {}& a+b+c+{} \\
&d+e+f+g
\end{aligned} \]
\paragraph{公式组}
\begin{gather}
a = b+c+d \\
x = y+z
\end{gather}
\begin{align}
a &= b+c+d \\
x &= y+z
\end{align}

\subsection{分段函数}
\[ y = \begin{cases}
-x, \quad x\leq 0 \\
x, \quad x>0
\end{cases} \]

%------------------------------
\section{插入图片和表格}

\subsection{插入图片}
\includegraphics[width = .8\textwidth]{zhangyiju.jpg}

\subsection{插入表格}
\begin{tabular}{|l|c|r|}
 \hline
操作系统 & 发行版 & 编辑器\\
 \hline
Windows & MikTex & TexMakerX\\
 \hline
\end{tabular}
\subsection{浮动体}
\begin{figure}[htbp]
\centering
\includegraphics{yiju.png}
\caption{有图有真相}
\label{fig:myphoto}
\end{figure}

\end{document}
